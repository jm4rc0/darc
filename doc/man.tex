%darc, the Durham Adaptive optics Real-time Controller.
%Copyright (C) 2010 Alastair Basden.

%This program is free software: you can redistribute it and/or modify
%it under the terms of the GNU Affero General Public License as
%published by the Free Software Foundation, either version 3 of the
%License, or (at your option) any later version.

%This program is distributed in the hope that it will be useful,
%but WITHOUT ANY WARRANTY; without even the implied warranty of
%MERCHANTABILITY or FITNESS FOR A PARTICULAR PURPOSE.  See the
%GNU Affero General Public License for more details.

%You should have received a copy of the GNU Affero General Public License
%along with this program.  If not, see <http://www.gnu.org/licenses/>.

\documentclass[a4,10pt]{article}
\usepackage{amsmath}
\newcommand{\ignore}[1]{}
\title{The Durham RTC}
\author{Alastair Basden}
\begin{document}
\maketitle
\section{Quickstart guide}
%darc, the Durham Adaptive optics Real-time Controller.
%Copyright (C) 2010 Alastair Basden.

%This program is free software: you can redistribute it and/or modify
%it under the terms of the GNU Affero General Public License as
%published by the Free Software Foundation, either version 3 of the
%License, or (at your option) any later version.

%This program is distributed in the hope that it will be useful,
%but WITHOUT ANY WARRANTY; without even the implied warranty of
%MERCHANTABILITY or FITNESS FOR A PARTICULAR PURPOSE.  See the
%GNU Affero General Public License for more details.

%You should have received a copy of the GNU Affero General Public License
%along with this program.  If not, see <http://www.gnu.org/licenses/>.

\begin{verbatim}
QUICKSTART for Paris, Oct 2009

To get the system into a working 2 camera mode:

Plug monitor and keyboard into telemetry server (small HP machine)
Network cable (motherboard) to your external network.
Network cable (PCI) to the local switch.
You will need to configure this machine for external network access.


Boot the telemetry server (small HP machine).
Wait until it has booted (5 minutes to allow crontab time to run).
It logs on automatically.
Open a terminal.
cd /Canary/rtc
./recvStream.py


Plug Network cable on camera PC (black dell machine) into the switch.
Plug the camera fibre into the right hand port.
Plug the camera trigger for the 2 cameras into the cameras and into
the two right hand BNC connectors.
Boot the camera PC (black Dell machine)

Plug network cable on RTC PC (mac pro) into port 1, and into the switch.
Plug the camera fibre into port 0 (right hand port).

Boot the RTC (mac pro).
Log into the RTC as canary twice (ssh from the telemetry server: ssh -X rtc ).
In each, type:
source /Canary/etc/rtc.bashrc
In one type:
cd rtc
./control.py

In the other type:
cd rtc
./startStreams.py -h192.168.0.1

Note - eventually, this should be self-starting, but when we changed
the network config just before shipping it stopped working, so please
do it manually for now.


On the telemetry server, 
open a terminal and type:
cd /Canary/rtc
./rtcgui.py


Make sure that you have a suitable config file (e.g. config2camera.py
or configshakti.py) in /Canary/rtc/ on the telemetry server and in
/home/canary/rtc/ on the rtc machine (mac pro).  

This is currently a bug, which will be fixed (we didn't want to fix it
in the hour before shipping!).

On the GUI, click the start button.  Choose the config file
config2camera.py.  The RTC should then start running.

If working in 1 camera mode, use configshakti.py.

Go to the streams tab, and turn on some streams.  
For example put 100 100 100 in the 3 entry boxes next to rtcPxlBuf.
This sets the decimate rates (for this GUI, out of the RTC between RTC
and tememetry server).
Do this for rtcPxlBuf

Click the spawn plot button.
Select the rtcPxlBuf button in the subscribe too... window, then close
the subscribe too... window.

This will then start showing camera pixels.
To view as an image, type into the text entry box in the bottom left
of the plot:
data.shape=128,256
and then click elsewhere on the screen (so that
the text entry looses focus).
This will show interleaved camera data.
To view from just one camera do either:
data.shape=128,256
data=data[:,::2]
or
data.shape=128,256
data=data[:,1::2]

Any valid python code can be inserted here.  It can be used for
implementing overlays or anything else you wish.

Note, to avoid having to do this, you can set up some plots, and save
them to a configuration file.  Some examples have been done for you -
run the GUI with:
./rtcgui.py initPhaseA.py
A list of available plots then appears in the GUI.
(some of these will work with 2 camera mode, some with 1 camera mode).
These can then be toggled to display or remove plots.

Feel free to edit initPhaseA.py as you wish.

To run some test scripts (some examples of how to do things with the camera):
cd /Canary/local/scripts/test
and look at these scripts.

Note - if any of these scripts require rtc output (pixels/slopes etc),
then the streams need to be turned on in the GUI first, as the script
facility to turn streams on hasn't been implemented yet.


\end{verbatim}



\section{RTC overview}
The RTC contains several main components:
\begin{enumerate}
\item Real-time core (darcmain)
\item Parameter and telemetry (output) buffer server (control.py)
\item Telemetry stream producer (startStreams.py)
\item Interface to telemetry server (recvStream.py)
\item Graphical interface (Python, rtcgui.py)
\item Telemetry server
\item Cameras
\item Scripts 
\item RTC dynamic libraries
\item Command line interfaces
\end{enumerate}

The real-time core initialises in shared memory the circular data
buffers (currently raw pixel, calibrated pixel, slopes, mirror
commands and status), and two parameter buffers which are used for
double-buffered operation.  The control server is used to set the
parameters in the core and (in non-telemetry mode) to retrieve data from the circular buffers.
The control server must have access to the same memory as the core
(i.e.\ must run on the same machine).  However, it can be accessed by
clients over a CORBA interface.  The GUI is one such client, and the
principle tool used by the operator.  The GUI allows the operator to
alter parameters, checking them for validity, to specify the
decimation rate at which data should be received from the control
server, and perform operations such as generating a poke matrix.

Telemetry from the real-time core is sent to the telemetry server
using a stream producer which reads the RTC circular buffers, and
sends the data over a socket to a receiving program, which then
converts to CORBA objects before passing to the telemetry server.

The internals of the real-time core are the most important part of the
control system, and so are now discussed further.

\ignore{
To run the RTC, you should use:
control.py  \#On the real-time computer
python recvStream.py \#Preferably on the telemetry server computer
python startStreams.py \#On the real-time computer
rtcgui.py \#Wherever you like.
}

You need the CORBA name server running (omniNames), with an
environment variable pointing to it and putting the software on the
python path:
\begin{verbatim}
export ORBInitRef="NameService=corbaname::omniHost.dur.ac.uk"
export PYTHONPATH=/Canary/src:\$PYTHONPATH
\end{verbatim}

\subsection{Running}
The telemetry server should already be running.

There are multiple ways of running the RTC depending on what type of
control you want, and what you want to happen with telemetry.

One way of doing this is with:

\begin{verbatim}
control.py 
startStreams.py -hIP.addr.of.telemetryServer.
\end{verbatim}


The recvStream.py should be running on the telemetry server.
Use the GUI (or a script) to initialise the RTC.
Alternatively, control.py config.py can be used, where config.py is
the name of your configuration file which will initialise the RTC and
start it running.

\subsection{Making}
The software is made in a few stages.  The first is to make the shared
object library for the mirror you wish to use (try make mirror for a
dummy version, or ``make libmirrorSL240.so'' for one using the SL240 card).
Then, make the shared library for the camera you wish to use (make
camera for a dummy version or ``make libSL240Int32cam.so'' for one that works
with the current SL240 interface).  Then, make the RTC itself, with
``make''.  The RTC is then called ``darcmain''.  It is installed with
``make install'' in /Canary or /rtc or whatever is specified in the Makefile.

You may also want to make libraries for reconstruction
(e.g.\ libreconmvm.so), figure sensing (e.g.\ libfigureSL240.so), the
WPU (e.g.\ libsl240centroider.so), but that is entirely dependent on
what you want to do.

Note: Intel processors perform a house-keeping task about once per second
which should be switched off using a linux kernel patch.


\subsection{Command interface}
A couple of programs are available which can be used in scripting.  

A RTCS\_run command (depreciated) allows any parameter to be set in the RTC
parameter buffer from the command line.  This is the recommended way
of simple scripting.  Additional commands to be called, for example to
produce a reconstruction matrix, etc.

The GUI allows parameters to be set, and allows scripting to be
carried out, e.g.\ interaction matrix generation.

For simple telemetry use TelemetryServer\_readNext or
TelemetryServer\_readNext\_Generic which will read a given number of
frames and write to a FITS file.

You can also implement your own scripting in python by creating a
controlCorba.controlClient() object.

A darctalk command allows getting and setting of parameters, and
reading of telemetry streams.

A darcmagic command implements the darctalk command set and additional
functionality, such as obtaining averaged images and slopes (averaged
in the RTC to guaranteed contiguous), and poking.  It is also slightly
more intuitive and requires less typing. 

\subsection{The real-time core}

The architecture of the real-time core is descibed here.

The work of processing the data frames is divided between threads,
with a user-specified number of threads for each frame grabber.  This
means that there will be at least as many threads as there are frame
grabbers, since one thread cannot process data from two frame
grabbers.  Here, a frame grabber refers to a piece of hardware which
is used to transfer data from one or more CCD cameras into a
contiguous memory region of PC main memory.  A WPU for example would
fall under this definition.  Henceforth in this document, we assume
that a frame grabber is responsible for one camera, however, this does
not necessarily need to be the case.

The decision to restrict each thread to it's own camera was taken
because the method by which pixels are waited for is not known.  This
may be a blocking call, in which case, if a single thread could serve
multiple cameras, it may be blocked waiting for pixels from one, when
in fact, pixels from all others had arrived.  Time would therefore be
wasted.  It will also allow the RTC to be altered in the future to
allow for cameras with different frame rates (this currently isn't
possible as it is not a requirement of CANARY, though the change
should be easy).

Each thread (one or more per camera) then processes one sub-aperture
at a time (Shack-Hartmann), until all have been processed, and the
wavefront reconstruction has been completed.  By processing on a
sub-aperture basis, latency is reduced, as the processing of a given
frame can start as soon as the pixels for that sub-aperture have
arrived, rather than waiting for the whole frame to arrive.  Each
thread takes pixels, calibrates them using provided background maps,
flat field images, etc.\ performes the slope estimation, and then uses
this to do as much wavefront reconstruction as possible (i.e.\ a
vector-vector multiplication).  

All sub-apertures of a given frame must have been processed before the
next frame is commenced.  Posix mutexes and condition variables are
used as thread synchronisation primatives.  Once the last sub-aperture
has been processed, the wavefront reconstruction will be complete.
Post-processing then commences in a separate global thread (actuator
clipping, sending the DM, etc.) while the sub-aperture processing of
the next frame starts (once pixels have started to arrive).

You are probably advised keeping the number of threads roughly equal to the
number of processors, and bearing in mind that one thread is used for
global post-processing.

\subsubsection{Parameter buffers}
The parameter buffer used to control the RTC is double buffered.  When
a buffer switch is requested, this is carried out by all threads
before the start of a frame.  The processing time required for a
switch is fairly minimal, and does not significantly increase the
pipeline latency.  There are plans to further reduce this switch processing time when time allows.

Most illegal parameters are caught by the GUI.  Uncaught illegal
parameters may have undesired effects, and once known about, the GUI
should be updated to catch them.


\subsubsection{Reconstruction}
There is a modular reconstructor interface, which allows different
reconstructor algorithms to be developed and experimented with
(without even stopping the RTC).  The standard one is libreconmvm.so,
which performs a simple matrix vector multiplication.  This can have
several different modes of operation, depending on whether you are
running in open or closed loop, and how much sophistication you
require.  


\subsubsection{Kalman filtering}
An open-loop implementation of Kalman filtering from SPIE Glasgow has
been implemented, assuming the latencies presented in that paper.  The
number of phase points is user defined, and the various matricees
required should be of the appropriate shape.  Three phase instances
are kept (next, this, previous), following the example in the paper.
These phases can then be passed either to the DMC (when implemented),
or used to compute actuator values.  If Kalman filtering is not
required, a standard matrix-vector implementation is also present,
using a user supplied control matrix.  This is available in the
libreconKalman.so library.





\subsubsection{Actuator gains}
A unique gain for each actuator can be supplied, rather than a
constant gain for each mirror.  This allows greater control and
flexibility.

\subsubsection{WPU interation}
The RTC can be set to receive either slopes or images from the WPU.

\subsubsection{Initialisation}
The RTC creates the shared memory parameter buffers, and then waits
for a client to write valid parameters to them.  After this, it then
initialises based on these values, and commences processing.
Currently, the number of threads and cameras cannot be altered without
restarting the RTC.  If there is a need for this change to occur while
on line, the code could be altered without too much difficulty.

\subsection{Interfaces}
Camera and mirror interfaces are modular.  By changing the shared
object library used, you can use different cameras
and mirrors without having to restart or recompile the RTC.  

The reconstructor and figure sensing interfaces are also modular.


\section{RTC configuration}
The RTC is configured on initialisation and at any time during
operation by writing to a parameter buffer in shared memory.  A client
will do this via the control object.

The list of parameters used is described here.

During initialisation, all of these parameters must be present in the
buffer.  Initialisation is typically carried out using a configuration
file, e.g.\ configshakti.py or config2camera.py.  

\subsection{ncam}
This is the number of frame grabber objects in the system (note, not
the number of cameras necessarily).  For CANARY phase 0, this will be
1, while for CANARY phase A, this will be 2 (3 WFS, 1 truth sensor,
combined into 2 pixel streams).

Type int32

\subsection{nacts}
The total number of actuators for the system.

Type int32

\subsection{nsubx}
An array of length ncam, specifying the number of sub-apertures in a
horizontal direction with an entry for each frame grabber.

Type array,int32,shape=ncam

\subsection{nsuby}
An array of length ncam, specifying the number of sub-apertures in a
vertical direction with an entry for each frame grabber.

Type array,int32,shape=ncam

\subsection{npxlx}
An array of length ncam, specifying the number of pixels in a
horizontal direction with an entry for each frame grabber.

Type array,int32,shape=ncam

\subsection{npxly}
An array of length ncam, specifying the number of pixels in a
vertical direction with an entry for each frame grabber.

Type array,int32,shape=ncam

\subsection{kalmanPhaseSize}
The size of the kalman output vector used with the kalman
reconstructor interface.

\subsection{kalmanUsed}
DEPRECIATED Flag specifying whether Kalman filtering is used.

\subsection{refCentroids}
An array of size equal to twice the total number of valid sub-apertures.  Specify the
reference slope measurements (or None) which are subtracted from slope
measurements.

Type None

Type array,float32,shape=nValidSubaps

\subsection{subapLocation}
An important structure, specifying the layout of sub-apertures for the
system.  This is an array of shape (nsubaps,6) where nsubaps is the
total number of sub-apertures (valid and invalid) for the system.
This can be computed by (nsubx * nsuby).sum().  For each sub-aperture
there are then 6 values,
$[y_{start},y_{end},y_{step},x_{start},x_{end},x_{step}]$.  These are
the pixel values that the sub-aperture starts and finishes at, and the
number of rows or columns to step.  With interleaved pixel data from 2
cameras, $x_{step}$ should be 2.  

For unused sub-apertures, the step value should be zero.

Type array,int32,shape=totSubaps,6

\subsection{bgImage}
The background image (or None).  Size should be equal to the total
number of pixels from all cameras, and pixel ordering should be as
received by the RTC.

Type None

Type array,float32,shape=npxls

\subsection{darkNoise}
The dark map image (or None)

Type None

Type array,float32,shape=npxls

\subsection{flatField}
The flat field image (or None).

Type None

Type array,float32,shape=npxls

\subsection{thresholdAlgorithm}
The thresholding algorithm to be used.  If equal to 1, 
values less than the threshold are set to zero.  If equal to 2, the
threshold is subtracted from all values, and negative values are set
to zero.  Other values mean no thresholding.

Type int32

\subsection{thresholdValue}
The value use for thresholding.  This can be either a single value, or
an array of size equal to the total number of sub-apertures, with a
value per sub-aperture, or a value per pixel.

Type float

Type array,float32,shape=totSubaps



\subsection{powerFactor}
A value to raise pixel values to the power of.

Type float

\subsection{centroidWeighting}
Not currently used.

\subsection{windowMode}
Should be a string either ``basic'' or ``adaptive'' to switch between
basic and adaptive windowing mode.

Type string

\subsection{kalmanReset}
Reset the internal Kalman state.  Remains set/unset until changed by
user.  Used when using the kalman reconstruction interface.

\subsection{usingDMC}
DEPRECIATED A flag to specify whether the DMC will accept actuator values (if 0),
or a Kalman state vector (if 1).  Only used if kalmanUsed flag is set.

\subsection{kalmanHinfT}
A matrix for Kalman filtering only used with the kalman reconstruction interface

\subsection{kalmanHinfDM}
A matrix for Kalman filtering only used with the kalman reconstruction interface

\subsection{kalmanAtur}
A matrix for Kalman filtering only used with the kalman reconstruction interface

\subsection{kalmanInvN}
A matrix for Kalman filtering only used with the kalman reconstruction interface

\subsection{subapFlag}
An array of size equal to the total number of sub-apertures, with a
flag value for each, specifying whether this sub-aperture should be
used.

Type array,int32,shape=totSubaps

\subsection{go}
When set to zero, the RTC will stop and exit.

Type int32

\subsection{pxlCnt}
An array of size equal to the total number of sub-apertures.  The
entries are the total number of pixels that are required for each
sub-aperture before processing of this sub-aperture can commence.
Note this can be computed from subapLocation.

Type array,int32,shape=totSubaps

\subsection{centroidMode}
A text string, equal to one of ``WPU'', ``CoG'', ``Gaussian'',
``Correlation CoG'', ``Correlation gaussian''.  Gaussian modes are not
yet implemented.  WPU mode means that the RTC is expecting slopes from
the frame grabbers.  CoG and Correlation CoG modes apply centre of
gravity calculations or a correlation calculation for slope
calculation.

Type string


\subsection{gainReconmxT}
Used internally, the gain multiplied by the reconstructor matrix.
Should not be supplied by the user.  Used only with the matrix-vector
reconstruction interface.

\subsection{reconstructMode}
A string with value equal to one of ``simple'', ``truth'', ``open'' or
``offset''.  This is used when kalmanUsed is not set.

If simple, the new DM command is zeroed.  

If truth, the new DM command is updated with the latest sent,
optionally multiplied by a decay factor value.

If open, the new DM command is set to $V_0 + gain \times (E \cdot
d_{latest})$ where E is the E matrix, and $d_{latest}$ is the latest DM
command sent.

If offset, the new DM command is set to $V_0$.

After this, the slopes are computed, dotted with the reconstruction
matrix, and the result added to the new DM command, which is then sent
to the mirror (after clipping etc.).

Used only with the matrix-vector
reconstruction interface.

Type string

\subsection{gainE}
Used internally, the gain multiplied by the E matrix.  Used only with the matrix-vector
reconstruction interface.


\subsection{v0}
Initial voltages (actually DAC values) used, one per actuator.

Type array,float32,shape=nacts

\subsection{bleedGain}
The gain used for the DM bleed algorithm.

Type float32

\subsection{actMax}
The maximum allowed actuator value, per actuator.

Type array,uint16,shape=nacts

\subsection{actMin}
The minimum allowed actuator value, per actuator.

Type array,uint16,shape=nacts

\subsection{pause}
A flag, which if set pauses the RTC (though camera frames are still
read).  No image calibration, slope calculation or DM vector
computation is carried out.

Type int32

\subsection{printTime}
Primarily for debugging, if set, the RTC will print the frame time.

Type int32

\subsection{ncamThreads}
An array of length ncam, with a value for each frame grabber
specifying the number of threads for each.  Use this to fine tune RTC
performance.

Type array,int32,shape=ncam

\subsection{switchRequested}
When set, the RTC will switch parameter buffers.

Type int32

\subsection{actuators}
A user defined array of actuators, of size equal to the number of
actuators times some value.  These will be placed on the mirror as
sets of actuators in turn, either replacing RTC
computed values, or added to them (possibly with a mask).

Type None

Type array,float32,shape=N,nacts where N is the number of entries in
your sequence (can be 1)

\subsection{fakeCCDImage}
For testing, if specified, a fake CCD image.

Type None

Type array,int32,shape=npxls

\subsection{threadAffinity}
An array of length equal to 1 + ncamThreads.sum().  A value for each
thread (the first being the post-processing thread) defining a CPU
mask for where the threads should be allowed to run.  Use this to fine
tune the RTC.

Type None

Type array,int32,nthreads+1

\subsection{threadPriority}
An array of length equal to 1 + ncamThreads.sum().  A value for each
thread (the first being the post-processing thread) defining the
thread priority.  Use this to fine tune the RTC.  It the RTC is not
run with the correct permissions, this will be ignored (will print
out a warning message).  Typically, the RTC should be run as root to
allow this to be used.

Type None

Type array,int32,nthreads+1



\subsection{delay}
An optional delay in microseconds, for debugging purposes to slow the
RTC down when no real cameras are attached.

\subsection{maxClipped}
The maximum number of actuators allowed to be clipped before an error
is raised.

\subsection{clearErrors}
Used by the error handler in the GUI to acknowledge errors.

\subsection{camerasOpen}
A flag used to open the camera shared object library
(librtccamera.so).  

\subsection{camerasFraming}
A flag used to start the cameras framing, i.e.\ start using camera
data.

\subsection{cameraParams}
An array of 32 bit integers which is passed into the camera object to
configure it.  The values in this array depend on which camera library
is used.  For example, if using the FITS file reader camera object,
this is just the filename
(e.g. numpy.fromstring(``shimage.fits'',dtype=numpy.int32)), while for
the SL240 interface, this has 5 x ncam values, equal to [blocksize,
timeout, port, affinity, priority] for each camera, where blocksize is
the number of pixels that should be read as a block (usually a few
rows at a time).  Timeout is the time to wait if data is not received
before raising an error and trying again.  Port is the SL240 fibre
port, numbered from 0 being closest to the motherboard for Curtiss
Wright cards.  Affinity is the thread CPU affinity and priority is the
thread priority for this frame grabber.

\subsection{cameraName}
The name of the camera object expected from the shared object library.
This will raise an error if not actually in the camera library.

\subsection{mirrorOpen}
A flag specifying whether the mirror library is in use or not.

\subsection{mirrorName}
The name of the mirror object expected from the shared object library.
This will raise an error if not actually in the mirror library.

\subsection{frameno}
Updated by the RTC, the frame number at which a buffer switch last
occurred.

\subsection{frameTime}
Updated by the RTC, the time at which a buffer switch last occurred.

\subsection{adaptiveWinGain}
A gain factor to use when adaptive windowing.

\subsection{adaptiveWinGroup}
When using global adaptive windowing mode, it is possible (optional)
to specify a set of groups to which individual sub-apertures are
assigned.  This way, if you have multiple cameras, they can each have
their own global adaptive windowing.  Or, if you have multiple
wavefront sensors on a single camera, you can again give different
adaptive windowing to each wavefront sensor.  This is None or an int32
array of size equal to the number of valid sub-apertures.  The entries
then correspond to each sub-aperture, and are the group number to
which this sub-aperture is assigned, starting from group 0.  The total
number of groups is unrestricted.  If you set this value to
range(totUsedSubaps) then you will be back to a adaptive windowing on a
per-subaperture basis.

\subsection{correlationThresholdType}
Used in correlation centroiding mode.
If zero or one, the correlation threshold value is set to
the user defined value.

If two or three, the correlation threshold is set to the maximum
correlation value multiplied by the user defined correlation threshold
value.

If zero or two, values below the correlation threshold are set to
zero.  If one or three, the correlation threshold is subtracted from
values, and then any negative values set to zero.

\subsection{correlationThreshold}
Either a threshold value or fraction of maximum value, depending on
correlationThresholdType.

\subsection{fftCorrelationPattern}
The FFT of the correlation pattern used in correlation centroiding, in
the correct format.  Use correlation.py to calculate this.  The GUI
will to this for you.

\subsection{nsubapsTogether}
The number of sub-apertures to be processed in a row by a single
thread.  Use this to tune RTC performance.

\subsection{nsteps}
If greater than zero, the number of iterations to run the RTC for
before pausing.  Used in debugging.  Set to -1 to run indefinitely.

\subsection{closeLoop}
A flag, whether the loop should be closed (values sent to DM) or not.

\subsection{mirrorParams}
The parameters supplied to the mirror library, an array of 32 bit
integers, contents depending on the mirror library.  For the SL240
mirror library, there are 4 values, being the timeout, port, thread
CPU affinity and thread priority.

\subsection{addActuators}
Whether user defined actuators should replace (if 0) RTC computed
values, or add (if 1) to them.

\subsection{actSequence}
When actuators are specified, actSequence defines the number of times
each set of actuators should be placed on the mirror.  This size of
this array should be equal to the number of sets of actuators
specified.

\subsection{recordCents}
Used to compute an interaction matrix or reference slopes.  If set,
slope measurements are recorded and avereged while an actuator
sequence is played.

\subsection{pxlWeight}
The pixel weighting factor, an array of values to multiply pixels by.
This allows weighted centre of gravity algorithms to be used.

Image calibration is performed using
\begin{enumerate}
\item Dark noise subtraction
\item Flat field multiplication
\item Background subtraction
\item Application of threshold
\item Multiplication by pixel weighing
\item Application of power factor.
\end{enumerate}

\subsection{averageImg}
Used by GUI/in scripting, The number of frames to average calibrated
images for before publishing the result to the generic stream.

\subsection{centroidersOpen}
In WPU mode, a flag determining whether the centroid frame grabber
shared object library is in use.

\subsection{centroidersFraming}
In WPU mode, to determine whether the centroid frame grabber should be
reading data.

\subsection{centroidersParams}
c.f.\ cameraParams

\subsection{centroidersName}
c.f.\ cameraName

\subsection{actuatorMask}
A mask of size equal to the number of actuators to specify which user
defined actuators should be used.

\subsection{dmDescription}
Currently used only by the GUI, for DM control.

\subsection{averageCent}
As with averageImg, equal to the number of frames to average centroids
too before publishing the result once to the generic stream.

\subsection{calmult}
Used internally
\subsection{calsub}
Used internally
\subsection{calthr}
Used internally
\subsection{centCalSteps}
An array of 32-bit floating point, size equal to the number of slope
measurements multiplied by the number of steps to be used in slope
linearisation\ref{sect:linearisation}.  These values are interpolated
to get the linearised slope measurement.

\subsection{centCalData}
An array of 32-bit floating point, size equal to the number of slope
measurements multiplied by the total number of steps to be used in
slope linearisation\ref{sect:linearisation}.  These values are the
non-linear slope measurements that correspond to the linearised
measurement given by centCalSteps.  If an un-linearised slope
measurement has a value equal to an entry in centCalData, then the
corresponding entry in centCalSteps will be returned.

\subsection{centCalBounds}
An array of 32-bit integer, size equal to twice the number of slope
measurements.  The values here are the lower and upper bounds for each
used sub-aperture, in which the linearisation steps and data are valid.

\subsubsection{Using centroid calibration}
Note - this is untested, but should work.  To use centroid
calibration, first place a range of tilts on a tip-tilt mirror that
cause the spots to move across the range of the WFS.  Then, record
slope measurements while this is going on.  Assuming that we put N
steps on the mirror, then the slope measurements can form an array of
shape (ncents, N), note that the measurements for one sub-aperture
should be contiguous in memory.  This array is centCalData.  Have a
look at the centCalData entries, and where they are distinctly
non-linear (e.g.\ for spots near the edges of sub-apertures) set the
bounds to these points, e.g.\ if below centCalData[X,i] is distinctly
non-linear (for sub-aperture measurememt X) then centCalBounds[X,0]
should be set to i, and similarly if above centCalData[X,j] is
distinctly non-linear, then centCalBounds[X,1] should be set to j.
Then, create an array centCalSteps, of the same size of centCalData
but that contains the tip-tilt values (i.e.\ linear), scaled such that
they have the same range as the centCalData entries for each
sub-aperture at the points centCalBounds[X,0] and centCalBounds[X,1].
Then put these into the RTCS and enjoy linearisation.


\subsection{comment}
An optional comment added to the buffer.  It would be hoped that
a sensible comment can be added when the user updates the buffer.
e.g.\ to specify the change they made.  It should also be noted that
each parameter in the buffer can also have an attached comment.

\subsection{rmx}
The reconstructor matrix, shape nacts, ncents.  Used only when the
matrix-vector reconstruction interface is used.

\subsection{E}
The E matrix.  Used only when the
matrix-vector reconstruction interface is used.

\subsection{gain}
The actuator gain, and array of size nacts, specifying the gain for
each actuator.  


\subsection{figureOpen}
Whether the figure sensor library is open or not (user and RTC
settable).

\subsection{figureName}
The name of the figure sensor library, if using RTC in a figure
sensing mode.

\subsection{figureParams}
Parameters to be passed to the figure sensor library at open time.

\subsection{reconName}
Name of the reconstructor library.

\subsection{fluxThreshold}
The minimum flux allowed for slopes to be non-zero.  A single value,
or a value per sub-aperture.

\subsection{useBrightest}
Zero, or the number of brightest pixels to use for the slope
calculation.

\subsection{figureGain}
Gain used when figure sensing (ie required actuators are multiplied by
this value before being used).  One value per actuator.

\subsection{decayFactor}
The decay factor to be used when using libreconmvm.so and when
reconMode==''truth''.  This can be None, or defines an array of values
of size nacts (float32), which are multiplied with the actuator values from
previous calculation before being added to the newly computed values.

\subsection{reconlibOpen}
Whether the reconstructor library is open (set to 1 to open it, and if
it is still 1 once the RTC has tried to open it, you know that the
open was successful).

\subsection{maxAdapOffset}
The maximum shift of a sub-aperture allowed when using adaptive
windowing (or zero to allow unrestricted movement).  A single global
value or an array with an entry for each active sub-aperture.

\subsection{bufferOpen}
Whether the buffer interface library is open.

\subsection{bufferName}
Name of the buffer interface library to use, e.g.\ librtcbuffer.so

\subsection{bufferUseSeq}
Whether buffer sequencing should be active.

\section{The parameter update interface}
DARC has the ability to update any parameter on a frame by frame
basis, using a buffer shared library.  This library is dynamically
loaded into the RTC, and of course can be unloaded and changed by the
user to suit their requirements.  The standard buffer update library,
librtcbuffer.so, uses an entry in the parameter buffer called
``bufferSeq'', and interprets this as a sequence of data to be used by
the RTC.  The user could write an alternative version that accepts new
parameters via a socket or shared memory etc, and then load this into
the RTC.

The contents of the bufferSeq array are a series of headers and data
units.  The headers are composed of four 32 bit integers, with values
being: Header size (32), Number of entries in the data, Number of
iterations before moving onto the next set in the series, and the size
of the header plus the data (so the pointer to the next set in the
series).

Each data section is organised as a parameter buffer and interpreted
as such, with one caveat:  If the length of the comment for a
particular variable is zero, this means change the active parameter
contents, while if non-zero, this means change the inactive parameter
buffer contents.

Arrays can generally be written to the active buffer (so long as they
aren't used by the pre or post processing threads), while other values
should be placed into the inactive buffer, and then
``switchRequested'' set in the active buffer, so that the RTC does a
buffer swap and switches them into action.  Once the end of the series
of buffers is reached, the RTC wraps around back the the beginning (if
this isn't the desired behaviour, then ``bufferUseSeq'' can be unset
in the inactive buffer as the last step in the sequence).  

If you wish to change parameters on a frame by frame basis, and these
aren't arrays (so, you must initiate buffer swaps each time), then
when setting ``bufferUseSeq'' to one, to start the sequence, you
should make sure that the copy flag is set to zero, so that active
buffer contents are not copied into the inactive buffer (which by that
time may be active).  This is achieved for example using
{\rm darctalk set -name=bufferUseSeq -value=1 -copy=0}  or, using the
{\rm ctrl=controlCorba.controlClient()} object by using {\rm
  ctrl.set(``bufferUseSeq'',1,copy=0)}.  In this case, you should also
ensure that during the first iteration, the bufferUseSeq in the
inactive buffer is also set to 1.

You should take care not to do other RTC operations that will involve
a buffer copy or swap while the sequence is running.

Note, if you wish to have a sequence of actuators, this can be done
simply by specifying a larger ``actuators'' array.  This is a special
case, and the buffer update interface is not required.

To create a suitable ``bufferSeq'' array using python, the following
code will work (and will alternate the reference slopes used by the RTC).

\begin{verbatim}
import buffer
import controlCorba

bs=buffer.BufferSequence()
bs.add("refCentroids",numpy.ones((288,),"f"),0)
bs.add("refCentroids",numpy.ones((288,),"f")*0.5,1)
bufseq=bs.makeBuffer(2)

ctrl=controlCorba.controlClient()
ctrl.set("bufferSeq",bufseq)
ctrl.set("bufferUseSeq",1,copy=0)
\end{verbatim}

To create a suitable ``bufferSeq'' array for a non-array parameter,
you could use:
\begin{verbatim}
import buffer
import controlCorba

bs=buffer.BufferSequence()
bs.add("useBrightest",5,0,inactive=1)
bs.add("useBrightest",25,1,inactive=1)
bs.add("useBrightest",0,2,inactive=1)
bs.add("switchRequested",1,0)
bs.add("switchRequested",1,1)
bs.add("switchRequested",1,2)
bs.add("bufferUseSeq",1,0,inactive=1)
bufseq=bs.makeBuffer(3)

ctrl=controlCorba.controlClient()
ctrl.set("bufferSeq",bufseq)
ctrl.set("bufferUseSeq",1,copy=0)
\end{verbatim}

Awesome!

\section{RTC Performance}
The RTC has not yet been used with a real-time kernel.  However, this
should be easy to install and test if necessary (e.g. sudo apt-get
install linux-rt on ubuntu).  Installing with a real-time kernel may
be unnecessary (the Fedora 12 real-time kernel does not appear to boot).

\subsection{Jitter}
Worst case for phase0A and phaseC jutter is about 200 microseconds,
while RMS jitter is about 25 microseconds, timed in software using the
RTC timing functions.

\subsection{Thread priorities}
Jitter can be reduced and performance improved by changing thread
affinities and priorities.

For phase C, with 1 thread per camera, a good option is to use
threadAffinity=[15,1,2,4,8,16,32,64,128] and
threadPriority=[40,30,30,30,30,50,50,50,50].  This gives a small RMS jitter.
Note, the first entry in these lists are for the post-processing
thread.

The number of sub-apertures processed at a time by each thread can
also be changed to improve performance.


\subsection{GPU processors}
Tests have been carried out for timing of BLAS functions used within
the RTC, comparing the execution time on CPU and on GPU.  For the
array sizes that are likely to be used for CANARY, the GPU performs
worse than the CPU.  Therefore, the GPU will not be used (and is not
required) for CANARY.  It should be noted that these tests were only
carried out using the vendor supplied BLAS libraries, and that no
effort has been made to port more of the RTC to GPU, which may well
lead to a performance improvement, though would require significant
coding effort.


\subsection{Latency}
The RTC is able to run at about 1kHz for a phase C implementation with
Kalman filtering, which
has 8 cameras, all 128x128 pixels, and 4 with 8x8 sub-apertures, the
other 4 having 16x16 sub-apertures, and with 310 actuators.  To
achieve this performance, the system can be run as root (so that
thread priorities can be assigned), and thread affinities should be
set such that the pre/post process thread runs on same CPUs as the 8x8
sub-aperture threads, which are shared between 3 CPUs, and the 16x16
sub-aperture threads have their own CPUs.  This leaves one CPU spare
for the system for reducing jitter.  Without kalman filtering, it
manages about 2~kHz.

Phase 0 can run at well over 5~kHz (0.00015 seconds per frame).

At one point I did investigate whether it would be quicker to do the
partialReconstruct() method in blocks, using SGEMV.  However, this
didn't seem to make any speed difference, though probably would for larger systems.

\subsection{Camera interfaces}
\subsubsection{libsl240Int32cam.so}
This library is used for cameras sending data on sFPDP.  Camera params
should be an array of type int32, and size $5*n_{interface}+2$ where
$n_{interface}$ is the number of fibre channels (e.g.\ for CANARY, the
number of WPUs).  As an example, the following could be in your
configuration file.
\begin{verbatim}
cameraParams=numpy.zeros((5*ncam+2,),numpy.int32)
cameraParams[0::5]=128*8#blocksize
cameraParams[1::5]=1000#timeout/ms
cameraParams[2::5]=range(ncam)#port
cameraParams[3::5]=0xffff#thread affinity
cameraParams[4::5]=1#thread priority
cameraParams[-2]=0#resync mode flag
cameraParams[-1]=1#wpu correction mode flag
\end{verbatim}

\subsubsection{libcamfile.so}
This library is used to read camera data from a FITS file.
Camera params should be the filename of the FITS image file.
For example in a configuration file you could use:
\begin{verbatim}
cameraParams=numpy.fromstring("myfile.fits\0",dtype="i")#Note, the \0 is necessary to pad the string to a multiple of 4 bytes
\end{verbatim}
Using darctalk you can use darctalk set -name=cameraParams -string=myfile.fits

The FITS file should be of type int16.

\subsection{Scaling to EAGLE}
The idea here is that in the next few years, processors with large
numbers of cores will become available.  Therefore, since this design
is multi-threaded, we can estimate how many processors might be needed
for EAGLE (for example).
Phase 0 with 1 thread, and 8192 actuators runs at 1.6 kHz.
Phase 0 with 1 thread, and 4096 actuators runs at 2.7 kHz.
Phase 0 with 1 thread, and 2048 actuators runs at 3.8 kHz.
So, assuming linear now, 16k actuators 800Hz
32k actuators 400Hz
64k actuators 200Hz
This is 64 sub-apertures in this thread.
In total for eagle there are about 84000 subaps.
So, this would require about 1300 processing cores.
If 128 cores per processor, 10 processors needed for 200Hz
40 processors would give 800Hz.

Phase 0 with 64x64 pixels, 4x4 subaps gives 5kHz, so scales
proportionally to number of sub-apertures (all sub-apertures used)

Using WPU doesn't give much improvement since number of sub-apertures
here is small.  

\subsection{Palm 3000}
Using similar values to initialise the RTC as used for the Palm 3000
system on Palomar (cite SPIE vol 6272, Dekany et al), with 64x64
sub-apertures and 3566 active actuators, we are able to run at a frame
rate of 50~Hz.  Therefore, a system with about 40 such nodes, should
reach the desired frame rate of 2~kHz, with little extra development.
When using an FPGA based pre-processor to compute wavefront slopes
from pixel data (e.g.\ the WPU), we achieve a frame-rate of up to
1~kHz, requiring only two such nodes to reach the 2~kHz frame-rate.

The WPU could manage 2~kHz with a sub-microsecond latency (last camera
pixel in to last slope measurement out).

Such a system would greatly reduce the complexity of that which will
be used, including 24 FPGAs and 48 DSPs.

\subsection{Andor PCI}
With the Andor attached via PCI, a 500~Hz frame rate is achieved,
basically limited by the camera.  The number of missed frames is low,
typically one in 120000, i.e.\ every 4 minutes.  The RMS jitter in
frame time is also fairly low, about 200 microseconds (very rough
estimate from looking at a timing plot), and CPU usage is about
$60\%$.  It is thought that this jitter is due to the Andor API, but
may be due to the DMA.  Min-max jitter is about 500 microseconds.  

With camera input from SL240 cards, the full frame rate is also achieved.



\subsection{Mirrors}
With a socket server attached (mimicing the DMC), the frame rate was
maintained at that of the camera, i.e.\ about 500~Hz.  A true SL240
interface DMC also works at this frame rate.


\section{Kalman filtering}
We assume that:
\begin{multline}\\
X_n=(\phi_{n+1},\phi_{n},\phi_{n-1})\\
X_{n+1} = ((A_{tur}, 0, 0), (I, 0, 0), (0, I, 0)) X_n + (I, 0, 0)
V_{n+1}\\
S_n = D (0, 0, M_\beta^L) X_n + W_n\\
\hat{X}_{n/n} = \hat{X}_{n/n-1} + H_\infty (S_n -C\hat{X}_{n/n-1})\\
\hat{X}_{n+1/n} = A\hat{X}_{n/n}\\
\end{multline}
We send $X_{n/n}$ to the DMC or $N^{-1} \phi_{n+1}$ to the actuators
if not using the DMC.

So, the calculations that are actually done are:
\subsection{Initialisation}
At the start of the frame, the following operations are done:

If a reset is requested, $X_{pred}$ is set to zero.  Otherwise, the matrix
$H_{infDM}$ is dotted with $X_{pred}[2]$ and the result subtracted from
$X_{pred}[0]$.  Here, $X_{pred}$ is assumed to be a matrix of size
(3,kalmanPhaseSize).  Actually, it is implemented as an vector of size
$kalmanPhaseSize\times 3$, but reshaping it makes writing this manual easier.


\subsection{Computation}
During frame computation, the following operation is done:

For each slope measurement, we scale the correct column of $H_{inf}$ by the
slope value, adding the result into Xpred (a thread local copy).  Once
all slopes have been computed, the local $X_{pred}$s are added to the
global $X_{pred}$ (thread safe).  This is equivalent to doing $X_{pred}+=H_{inf}
\cdot slopes$.  

\subsection{Finalisation}
At the end of the frame, the following operations are done:
If the DMC is not being used, the dot product of InvN and Xpred is
computed, and the result put into dmCommand.

Then: 
\begin{multline}\\
Xpred[2]=Xpred[1],\\
Xpred[1]=Xpred[0],\\
Xpred[0]=Atur \cdot Xpred[0]\\
\end{multline}

\section{Slope linearisation}
\label{sect:linearisation}
The RTC can perform slope linearisation, using interpolation from
user-supplied data.  This is essential for good open-loop operation.

The user will calibrate the wavefront sensors by applying a number of
steps to a tip-tilt mirror.  For each step, the slope measurements are
stored.  Then, during operation, when a un-linearised slope
measurement has been computed, the stored slope measurements are used
to interpolate to give the value of the step that would give the
un-linearised slope.  The step values should be scaled such that if a
un-linearised measurement is outside the linearisation range, then the
un-linearised measurement is suitable for use.

During the linearisation calibration, for each sub-aperture, the lower
and upper index in the step and slope measurement arrays that can be
used can also be specified, which is useful for dealing with highly
non-linear curves (typically near the edge of the sub-apertures).  


\section{Error handling}
For the purposes of the RTC, we split errors into two types - single
and repeated.  

A single error may for example be a configuration problem.  A repeated
error is one which may keep on arising, for example too many actuators
clipped.  

Single errors should be sent to the GUI immediately.  Repeated errors
should only be sent on the first occurrence after a user clear has
occurred.  

For the purposes of errors, the circular buffer is used simply as a
pipe, in which to pipe the errors to control.py (a real pipe wasn't
appropriate because it may screw up the RTC if it got full).  The
control.py or telemetry server is then responsible for passing these
errors to clients (upon connection and when new ones occur), and for
deleting/clearing them when requested to do so.  If two clients try to
delete the same error message, then nothing bad occurs, except that if
the error reoccurs between deletions, the second one will also be
deleted, meaning the clients may not see it.  Currently, clients are
not notified if someone else has deleted errors.


\section{Telemetry streams}
Output from the RTC is in the form of telemetry streams.  There are,
for example, a raw pixel stream, a calibrated pixel stream, a slope
measurement stream, a actuator stream, a status stream, etc.

In the case that the available hardware is not powerful enough to
handle all of these streams at full rate, they can be decimated, or
switched off.

There are three points at which decimate values are set.  The first is
the decimate value for the RTC, $d_1$, meaning that the RTC will only
output data every $d_1$ frames.  The next is the decimate value into
the telemetry server, $d_2$ and the third is the value for each
client, $d_3$.  Each value is relative to source, rather than to the
previous decimate value.  The values $d_2$ and $d_3$ may be approximate, for
example, if $d_3=10$ while $d_1=d_2=100$, then client will get every
100 frames.
Likewise, if $d_3=100$ and $d_1=d_2=80$, then client will get
160 frames.  But if $d_3=100$ and $d_1=25$ and $d_2=50$ then the
client will get every 100 frames as requested.

\section{GUI}
The GUI can be used to control the RTC, and to view telemetry stream data.

Telemetry stream clients (e.g.\ the GUI) have a separate decimate
value from the RTC and telemetry stream decimator values.  The GUI can
spawn multiple plots, each subscribed to which ever streams are
required.  There is an ability to load and save groups of windows
(including their position and mangle command).  Graphical overlays are
also possible, e.g.\ for sub-aperture masks etc.

The GUI is run with rtcgui.py [init script.py]

\subsection{GUI Initialisation scripts}
Initialisation scripts can be used when starting the GUI.  These are
Python scripts, which are executed as if they are part of the GUI.  If
one is specified on the command line, this is executed first.  If it
sets ``self.execNext=1'' then the next config file is executed.  This
is ./.rtcguirc.py.  Likewise, if this specifies ``self.execNext=1''
then ~/.rtcguirc.py' is executed, and then (if ``self.execNext=1'' is
set) /etc/rtcguirc.py is executed.  This way, any number between 0-4
files can be executed, depending on whether they exist or not.  More
than one configuration file can be specified on the command line when
starting the GUI.

\subsection{Default plot configuration setup}
When a plot configuration file is specified (load plots button), it
gets added to the plot configuration list.  This is a list of toggle
buttons, which can be clicked to hide or show the associated plots.
This list can also be created by a config file, so that there are some
default entries, e.g.\ by putting
self.loadPlotConfig("plotrawpxls.xml") into a config file.

\subsection{Time-series plots}
The standard GUI can be used to plot a time series.
To do this, in the mangle part of a plot, you could put something
like:
\begin{verbatim}
store=makeArr(store,(1000,),"f")
store[:999]=store[1:]
store[999]=data[1]
data=store
\end{verbatim}
This would plot a history of data[1] (assuming len(data.shape)==1)

\subsection{Overlays}
Overlays can be done using the standard GUI.
Simply create an array of shape [x,y,4] called overlay in the mangle,
where the dimensions are for RGBA.  
e.g.
\begin{verbatim}
overlay=numpy.zeros((100,100,4),'f')
overlay[::10,:,::3]=1
overlay[:,::10,::3]=1
\end{verbatim}
would put a red grid on the image.  The shape of the overlay doesn't
need to be the same as that of the image.
In pixel plots, a centroid overlay can also be plotted by toggling the
GUI button.  This only appears when overlay is reshaped, and when
centroids are also being grabbed.

\subsubsection{Centroid spots}
\ignore{
Plotting overlay of centroid spots on pixel images:  e.g. for a single
camera, 128x128 pixels, use:
data.shape=128,128;overlay.shape=128,128,4
Can use spot module to change the shape of the spots...
e.g.
\begin{verbatim}
data.shape=128,128
overlay.shape=128,128,4;import spot;overlay=spot.expandPoints(overlay)
spot.expandPoints can take ``cross'' or ``circ'' as 2nd argument
(typ), and an odd number as 3rd argument, diam.  Alternatively, a spot
shape can be passed to sh, e.g.
spot.expandPoints(overlay,sh=numpy.ones((3,3,4))) to create a white square.

Or, you can set self.centOverlayPattern to be whatever you want,
currently using the command panel, though this will probably become a
config option.  Shape is x,y,4.  e.g.
self.centOverlayPattern=numpy.zeros((8,8,4),"f")
self.centOverlayPattern[:,:]=1
self.centOverlayPattern[2:6,2:6,2]=0
self.centOverlayPattern[0::7,0::7]=0
\end{verbatim}
to get fried eggs.
}

Plotting an overlay of centroid spots on pixel images is possible.
For a single camera, the following text can be placed into the mangle
part of the plot window (right click on a plot window to display this,
if it is not visible):
\begin{verbatim}
if streamTime["Cen"][0]!=streamTime["Sub"][0] or \
       streamTime["Cen"][0]!=streamTime["Cal"][0]:
 freeze=1
else:#all frame numbers equal, so plot...
 data=stream["Cal"]
 data.shape=128,128
 import overlayMaker
 if not store.has_key("pattern"):
  store["pattern"]=overlayMaker.makePattern("egg",8)
 pattern=store["pattern"]
 if store.has_key("overlay"):
  overlay=store["overlay"]
 else:
  overlay=None
 overlay=overlayMaker.createSubapCentOverlay(stream,subapLocation, \
                                       pattern=pattern,out=overlay)
 store["overlay"]=overlay
\end{verbatim}

Note, to do this, you must be subscribed to (or have been subscribed
to) rtcCentBuf, rtcSubLocBuf and rtcCalPxlBuf
The shape used for spots can be changed, by setting the pattern
argument.  The oversampling factor can also be changed.

See plotcalimg4cam.xml or similar to have a nice image display.

\subsection{Zernike decomposition}
To plot a zernike decomposition, you can use, in mangle of the mirror
buffer or actuator buffer:
\begin{verbatim}
import zernike
data=zernike.makeZernike(data[2:2+52],10,zernike.Pupil(8,4,0).fn)
\end{verbatim}
This would plot the first 10 zernikes in the case that data[2:2+52]
gives the 52 actuators that make up a given mirror, the mask described
by the pupil function.

\subsection{Poke matrix generation}
The GUI or CORBA interface can be used to generate a poke matrix (or
any other response matrix, for example reference slopes).

Creating a response matrix involves putting a set of actuators on the
mirror for a given number of iterations, and averaging the slope
measurements obtained over this time.  This is then repeated for
different actuator sets, and a reponse matrix built up.  Mirror
settling time can be taken account of using this technique.

See the script tag of the GUI for this.





\section{Windows}
The GUI now works with MS windows (tested with XP), provided the
correct libraries are installed.  These are:
\begin{itemize}
\item python
\item pygtk, which needs:
\item pycairo, pygobject, pygtk, glade, gtk-dev, gtk, all found from links
on the pygtk website.
\item numpy
\item matplotlib
\item msvcp71.dll (google this and you'll find it - if you have visual c++
installed, you will already have this library - it is required if
someone compiles code in debug mode, and it appears that matplotlib
requires this).
\end{itemize}
After setting paths correctly, rtcgui.py should then run.

Note, this has not been tested for a while, so some code changes may
be necessary.




\end{document}
